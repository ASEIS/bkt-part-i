
\section{Introduction}

Modeling attenuation effects due to internal friction in geomaterials is an important aspect in geophysical, seismological and geotechnical engineering applications. Ignoring these effects may lead to the inaccurate overestimation of the ground motion during induced or natural (earthquake) shaking. Attenuation effects are particularly relevant over large distances or in the presence of highly dissipative media. Mathematical and numerical representation of energy losses in wave propagation problems is relatively simple in the solution of the representation theorem or the wave equation in the frequency domain. However, introduction of attenuation in the time-domain poses difficulties in terms of accuracy, implementation, and computational efficiency.

Most of the attenuation models proposed in the literature rely on introducing anelasticity by means of viscoelastic models which can reproduce the rheological behavior of sediments and rock materials. Physically, these models can be interpreted as mechanical elements made of elastic springs and viscous dampers, or dashpots \citep[e.g.][]{Moczo_2005_GRL}. Mathematically, these elements or relaxation mechanisms may be represented by the introduction of additional memory variables representing the relaxation process \citep[e.g.][]{Day_1984_GJI, Day_2001_BSSA}. Both interpretation relate to the convolution integral in the stress-strain relationship of viscoelastic materials. The concept of memory variables arises from the need for substituting the convolution term with a formulation that can be implemented in time-stepped numerical methods without having to store and use the complete history of strains.

There are numerous implementations of realistic attenuation in the time domain, most of which follow the initial ideas put forward by \citet{Day_1984_GJI}, \citet{Emmerich_1987_G} and \citet{Carcione_1988_GJI}. 