% 
\begin{abstract}
% 
Appropriate modeling of attenuation effects due to internal friction in wave propagation problems is particularly relevant over large distances or in the presence of highly dissipative media. Different alternatives are considered in the literature to represent energy losses due to internal friction. Most make use of viscoelastic models with combinations of spring and dashpot elements whose properties are associated to the attenuation quality factor, $Q$, in the frequency domain. The dynamic behavior of the mechanical model can be expressed in terms of stresses and strains, or directly in terms of displacements, and then introduced into the formulation of the wave equation. The solution of the resulting anelastic wave propagation problem in time poses significant difficulties in terms of numerical accuracy and computational efficiency. The parameters of the model need to be calibrated for discrete values of $Q$, and this requires the construction of tables with values that are later interpolated. This limits the generality of numerical implementations, and affects their accuracy. In addition, the solution in time requires the use of additional variables. The greater the number of mechanical elements, the greater the computational cost. More so if the implementation is done in terms of stress and strain components, as opposed to displacements. We revise and improve a previously introduced model that uses displacements as primary variables and a reduced number of mechanical elements. We present new empirical equations to obtain the model parameters in terms of the target $Q$ function, for both constant and frequency dependent $Q$. Our formulation eliminates the need for interpolation and maintains an excellent level of accuracy at a low computational cost for a wide range of $Q$ values. We validate the formulation implementation against semi-analytical solutions for one- and three-dimensional idealized problems, and test its applicability in realistic earthquake ground motion simulation.\\
% 
\end{abstract}
% 